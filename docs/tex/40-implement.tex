\chapter{Технологическая часть}

В данном разделе будут описаны детали реализации и развёртывания ПО, а также приведены листинги кода.

\section{Средства реализации}

Для реализации статического сервера был выбран язык C в соответствии с заданием на курсовую работу. Для мультиплексирования клиентских соединений был выбран мультиплексор select, для параллелизации обработки запросов --- пул потоков.

Для контроля качества кода использовался отладчик использования памяти valgrind \cite{valgrind}.

\section{Детали реализации}

\subsection{Сервер}

В листингах \ref{lst:server1} и \ref{lst:server2} представлены функции, реализующие обработку клиентских запросов с использованием мультиплексора select.

\begin{lstlisting}[
	caption={Функция создания сервера},
	label=lst:server1,
	]
	int server_create(server_t *server) {
		server_t tmp_server = malloc(sizeof(struct server));
		if (tmp_server == NULL) {
			log_error("server_init malloc(): %s", strerror(errno));
			return errno;
		}
		
		if ((tmp_server->server_socket_fd = socket(AF_INET, SOCK_STREAM, 0)) 
			== 0) {
			log_error("socket(): %s", strerror(errno));
			free(*server);
			return EXIT_FAILURE;
		}
		
		int opt = IP_PMTUDISC_WANT;
		if (setsockopt(tmp_server->server_socket_fd, SOL_SOCKET, 
			SO_REUSEADDR | SO_REUSEPORT, &opt, sizeof(opt))) {
			log_error("setsockopt(): %s", strerror(errno));
			free(tmp_server);
			return EXIT_FAILURE;
		}
		
		tmp_server->is_running = false;
		*server = tmp_server;
		
		return EXIT_SUCCESS;
	}
\end{lstlisting}

\begin{lstlisting}[
	caption={Функция запуска сервера},
	label=lst:server2,
	]
	int server_run(server_t server, int port, int conn_queue_len, 
		void(*handle_request)(int)) {
		struct sockaddr_in address;
		address.sin_family = AF_INET;
		address.sin_addr.s_addr = INADDR_ANY;
		address.sin_port = htons(port);
		
		if (bind(server->server_socket_fd, (struct sockaddr*)&address, 
			sizeof(address)) == -1) {
			log_error("bind(): %s", strerror(errno));
			return errno;
		}
		
		if (listen(server->server_socket_fd, conn_queue_len) == -1) {
			log_error("listen(): %s", strerror(errno));
			return errno;
		}
		
		log_info("server started on port %d; wait for connections...", 
			port);
		
		server->is_running = true;
		fd_set client_fds;
		while (server->is_running) {
			FD_ZERO(&client_fds);
			FD_SET(server->server_socket_fd, &client_fds);
			
			if (select(server->server_socket_fd + 1, &client_fds, NULL, 
				NULL, NULL) == -1) {
				log_error("select(): %s", strerror(errno));
				return errno;
			}
			
			int client_socket_fd = -1;
			if (FD_ISSET(server->server_socket_fd, &client_fds)) {
				if ((client_socket_fd = accept(server->server_socket_fd, 
					(struct sockaddr*)NULL, NULL)) == -1) {
					log_error("accept(): %s", strerror(errno));
					return errno;
				}
				
				handle_request(client_socket_fd);
			}
		}
		
		return EXIT_SUCCESS;
	}
\end{lstlisting}

\subsection{Пул потоков}

В листингах \ref{lst:pool1}-\ref{lst:pool3} представлены функции, реализующие обработку клиентских запросов с использованием мультиплексора select.

\begin{lstlisting}[
	caption={Функция запуска пула потоков},
	label=lst:pool1,
	]
	int thread_pool_start(thread_pool_t pool, 
		void *(*worker_thread)(void *)) {
		for (int i = 0; i < pool->capacity; i++) {
			if (pthread_create(&pool->threads[i], NULL, worker_thread, pool) 
				!= 0) {
				log_error("pthread_create(): %s", strerror(errno));
				return errno;
			}
		}
		
		return EXIT_SUCCESS;
	}
\end{lstlisting}

\begin{lstlisting}[
	caption={Функция извлечения задачи из очереди},
	label=lst:pool2,
	]
	int thread_pool_take_task(thread_pool_task_t *task, thread_pool_t pool) 
	{
		log_debug("try to take task from pool...");
		pthread_mutex_lock(&queue_mutex);
		
		log_debug("wait for any task pool...");
		while (pool->size == 0) {
			pthread_cond_wait(&queue_cond, &queue_mutex);
		}
		
		thread_pool_task_t t = pool->tasks[--pool->size];
		
		log_debug("task taken from pool");
		pthread_cond_signal(&queue_cond);
		pthread_mutex_unlock(&queue_mutex);
		
		*task = t;
		
		return EXIT_SUCCESS;
	}
\end{lstlisting}

\begin{lstlisting}[
	caption={Функция добавления задачи в очередь на выполнение},
	label=lst:pool3,
	]
	int thread_pool_submit(thread_pool_t pool, thread_pool_task_t task) {
		log_debug("try to put task to pool...");
		pthread_mutex_lock(&queue_mutex);
		
		log_debug("wait for place in pool to put task...");
		while (pool->size >= pool->capacity) {
			pthread_cond_wait(&queue_cond, &queue_mutex);
		}
		
		pool->tasks[pool->size++] = task;
		
		log_debug("task added to pool");
		pthread_cond_signal(&queue_cond);
		pthread_mutex_unlock(&queue_mutex);
		
		return EXIT_SUCCESS;
	}
\end{lstlisting}



\section{Поддерживаемые запросы}

Разработанный веб-сервер обрабатывает запросы GET и HEAD. В первом случае клиент получает в теле ответа запрошенный файл, во втором --- только заголовки ответа Content-Type и Content-Length. Ниже перечислены поддерживаемые статусы ответов сервера.

\begin{itemize}[label*=---]
	\item 200 --- успешное завершение обработки запроса.
	\item 403 --- доступ к запрошенному файлу запрещён.
	\item 404 --- запрашиваемый файл не найден.
	\item 405 --- неподдерживаемый HTTP-метод (POST, PUT и т.д.).
	\item 500 --- внутренняя ошибка сервера.
	\item 501 --- запрошен неподдерживаемый тип файла.
\end{itemize}

Разработанный сервер поддерживает следующие форматы файлов (типы контента, Content-Type):

\begin{itemize}[label*=---]
	\item html (text/html);
	\item css (text/css);
	\item js (text/javascript);
	\item png (image/png);
	\item jpg (image/jpg);
	\item jpeg (image/jpe);
	\item gif (image/gif);
	\item svg (image/svg);
	\item swf (application/x-shockwave-flash);
	\item mp4 (application/x-shockwave-flash).
\end{itemize}



\section{Развёртывание}

Запуск разработанного приложения осуществлялся с помощью системы контейнеризации Docker \cite{docker}. Файл для сборки образа приложения представлен в листинге \ref{lst:deploy1}.

\begin{lstlisting}[
	caption={Dockerfile образа приложения},
	label=lst:deploy1,
	]
	FROM gcc:13.2.0-bookworm AS build
	WORKDIR /build
	RUN --mount=target=. \
	gcc -DLOG_USE_COLOR \
		-std=gnu99 -Wall -Wpedantic -Wextra -Wfloat-equal \
		-Wfloat-conversion -Wvla  \
		-static \
		-Iinc/app -Iinc/http -Ilib/fs -Ilib/log \
		-O2 -o /app  \
		src/main.c src/app/* src/http/* lib/fs/fs.c lib/log/log.c
	
	FROM scratch
	COPY --from=build /app /app
	EXPOSE 8080
	ENTRYPOINT ["/app"]
\end{lstlisting}

\clearpage
В листинге \ref{lst:deploy2} представлены команды для сборки образа приложения и запуска соответствующего контейнера.

\begin{lstlisting}[
	caption={Команды сборки образа приложения и запуска контейнера},
	label=lst:deploy2,
	]
	docker build -t static-server .
	docker run --name static-server --volume=./media:/tmp/static -d \
		-p 8080:8080 --cpus=8 static-server
\end{lstlisting}



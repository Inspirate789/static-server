\chapter{Исследовательская часть}

В данном разделе будет проведено нагрузочное тестирование разработанного сервера и сравнение его показателей производительности с NGINX.

\section{Описание проводимых измерений}

В качестве эталона для сравнения показателей производительности статического сервера был выбран NGINX \cite{nginx}, являющийся HTTP-сервером общего назначения. Конфигурация NGINX приведена в листинге \ref{lst:nginx1}.

\begin{lstlisting}[
	caption={Конфигурация NGINX},
	label=lst:nginx1,
	]
	server {
		listen 80;
		location /tmp/static {
			alias /tmp/static;
			include /etc/nginx/mime.types;
			autoindex on;
			autoindex_exact_size off;
			autoindex_localtime on;
		}
	}
\end{lstlisting}

Файл для сборки образа приложения представлен в листинге \ref{lst:nginx2}.

\begin{lstlisting}[
	caption={Dockerfile образа NGINX},
	label=lst:nginx2,
	]
	FROM nginx:1.25.1
	COPY nginx.conf /etc/nginx/conf.d/default.conf
\end{lstlisting}

В листинге \ref{lst:nginx3} представлены команды для сборки образа сервера NGINX и запуска соответствующего контейнера.

\begin{lstlisting}[
	caption={Команды сборки образа NGINX и запуска контейнера},
	label=lst:nginx3,
	]
	docker build -t nginx-example .
	docker run --name static-server --volume=./media:/tmp/static -d \
		-p 8080:80 --cpus=8 nginx-example
\end{lstlisting}

Ниже приведены технические характеристики устройства, на котором выполнялось тестирование.

\begin{itemize}
	\item Операционная система: Debian 12, версия ядра 6.1.0-12-amd64.
	\item Объём оперативной памяти: 16 Гб.
	\item Процессор: Intel i5-9300H 2.4 ГГц \cite{intel}.
\end{itemize}

Тестирование проводилось на ноутбуке, включенном в сеть электропитания. Во время тестирования ноутбук был нагружен только встроенными приложениями окружения, а также непосредственно системой тестирования.

Для замеров метрик производительности сравниваемых веб-серверов использовалась утилита Apache Benchmark (ab). \cite{ab}



\section{Результаты измерений}

Результаты нагрузочного тестирования разработанного сервера и NGINX представлены в таблицах \ref{tab:1}-\ref{tab:3}. Для оценки производительности измерялось количество обработанных запросов в секунду (Requests Per Second, RPS).

\begin{table}[H]
	\centering
	\caption{RPS при обработке 10000 запросов на файл размером 2 Кб}
	\label{tab:1}
	\begin{tabular}{|c|c|c|}
		\hline
		Число клиентов & Разработанный сервер & NGINX \\ \hline
		1 & 414.97 & 406.74 \\ \hline
		100 & 1878.45 & 1823.23 \\ \hline
		1000 & 1798.11 & 1155.56 \\ \hline
	\end{tabular}
\end{table}

\begin{table}[H]
	\centering
	\caption{RPS при обработке 1000 запросов на файл размером 468 Кб}
	\label{tab:2}
	\begin{tabular}{|c|c|c|}
		\hline
		Число клиентов & Разработанный сервер & NGINX \\ \hline
		1 & 67.15 & 77.73 \\ \hline
		100 & 115.02 & 120.52 \\ \hline
		1000 & 113.96 & 122.20 \\ \hline
	\end{tabular}
\end{table}

\begin{table}[H]
	\centering
	\caption{RPS при обработке 10 запросов на файл размером 582 Мб}
	\label{tab:3}
	\begin{tabular}{|c|c|c|}
		\hline
		Число клиентов & Разработанный сервер & NGINX \\ \hline
		1 & 0.08 & 0.08 \\ \hline
		5 & 0.10 & 0.10 \\ \hline
		10 & 0.10 & 0.10 \\ \hline
	\end{tabular}
\end{table}



\section{Выводы}

При обработке относительно лёгких запросов RPS сравниваемых серверов не отличался более чем на 2\%, однако при увеличении числа конкурентных запросов производительность резко снизилась, став на 36\% ниже той, что показал разработанный сервер. При увеличении размера запрашиваемых файлов производительность NGINX оказалась выше на 4-15\%. При работе с относительно большими файлами оба сервера показали примерно одинаковую производительность и пропускную способность.

Данные результаты можно объяснить тем, что разработанный сервер имеет более простую архитектуру и компактную кодовую базу, чем NGINX. Также ввиду использования мультиплексора select, имеющего ограничение на количество открытых файловых дескрипторов (не более 1024), при одновременном обслуживании более 1000 клиентов разработанный сервер склонен к потере запросов. Для устранения данной проблемы следует использовать другой мультиплексор, например epoll.
